\section{Conclusion}

We presented a novel approach to generating coral reef island terrains by combining traditional procedural methods with deep learning techniques. We first developed a procedural generation algorithm capable of creating a wide variety of island terrains using top-view and profile-view sketches, wind deformation, subsidence, and coral reef growth simulation. By applying these methods, we were able to produce realistic terrains based on geological processes, capturing key features of coral reef islands such as beaches, lagoons, and reefs.

To further enhance flexibility and realism in the generation process, we incorporated a conditional Generative Adversarial Network (cGAN), using the pix2pix model to translate island feature label maps into height maps. The cGAN model helped to overcome some of the constraints inherent in the procedural algorithm, such as radial symmetry and fixed island positioning. using data augmentation techniques, we trained the cGAN on a synthetic dataset to produce varied and realistic island terrains.


% There are several directions for future research and improvements. One promising avenue is to incorporate the wind velocity field more directly into the cGAN training process, potentially as an additional input condition. This would allow the model to better capture wind-driven terrain features such as cliffs or other deformations influenced by wind patterns.

% Another area for exploration is improving user interaction during the terrain generation process. While the current model allows for rapid terrain generation, adding more options for users to interact with the cGAN, such as tweaking parameters like wind strength or island size, could enhance the flexibility of the system.

% Finally, further improvements could be made to the synthetic dataset. Incorporating more complex geological processes, such as wave erosion or tidal influences, could lead to even more realistic terrains. Additionally, refining the way islands are blended in multi-island samples, or adding more diverse input conditions (e.g., different geological settings), could help the model generalize better and produce more varied and dynamic landscapes.


% One possible future improvement could involve incorporating the wind velocity field into the cGAN training process. While the label map is the only input used in the current implementation, the wind field could be added as an additional condition. This would be especially useful if the initial algorithm were augmented to include wind-driven features, such as cliffs or specific terrain deformations influenced by wind patterns. Adding the wind field as an input could help the cGAN generate more realistic terrains that better reflect the influence of wind on the landscape.

% Additionally, further development could explore improving how multiple islands are combined in a single sample. For example, using blending techniques to handle overlapping regions could allow islands to be positioned closer together, enabling the generation of more complex archipelagos without sacrificing the integrity of the height field.

% Many other neural networks models could be exploited to increase the possibilities, such as newer variants of cGANs (\cite{Park2019}), or models with style transfer functionalities (\cite{Gatys2015,Zhu2020}) in order to change the overall aspect of a terrain (\cite{Perche2023a,Perche2023b}), use text-to-images models (\cite{Rombach2021,Radford2021}) to generate height fields from a verbal prompt, or super-resolution models (\cite{Dong2014}) to increase the definition of details in the final output (\cite{Guerin2016a}).