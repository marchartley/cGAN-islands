\section{Introduction}

Simulating the formation of coral reef islands presents significant challenges due to the complex interplay of geological, environmental, and biological factors (\cite{Hopley2014}). One major difficulty lies in capturing the long-term subsidence of volcanic islands, which occurs over millions of years, while simultaneously modeling the upward growth of coral reefs that rely on environmental conditions such as water depth, temperature, and sunlight. This combination of slow geological processes and dynamic biological growth is difficult to replicate in a computational model.

Additionally, the biological aspects of coral growth are inherently tied to environmental factors. Coral reefs grow only within a specific range of water depth and sunlight, and their growth patterns are affected by the health of the reef ecosystem and the availability of resources. Accurately modeling these biological dependencies in a procedural system is challenging, as these factors are numerous and difficult to generalize. Moreover, the scarcity of data available obstructs the global understanding of these biomes. In a recent high-resolution mapping of shallow coral reefs (\cite{Lyons2024}), researchers estimated the total surface area of this biome to cover less than 0.7\% of Earth's area, and more specifically that coral habitat represents less than 0.2\%.

Existing terrain generation methods, such as Perlin noise-based algorithms or uplift-erosion models, are often ill-suited for these processes. While they can generate natural-looking landscapes (such as alpine landscape, representing about a quarter of land area (\cite{Korner2014})), they do not account for the unique geological and biological interactions that govern coral reef island formation, thus missing coherency. Capturing these dynamics, while also providing user control during the modeling of a terrain, requires a balance between realism and procedural flexibility, allowing for both accurate computationally expensive simulation of natural processes and intuitive user control in interactive time.

% The formation of these islands involves processes at multiple scales, from the growth patterns of coral colonies to large-scale sediment transport, which are difficult to simulate directly. As a result, purely procedural or physics-based simulations can fail to produce convincing or diverse coral reef island landscapes. On the other hand, the use of deep learning methods are inoperable due to the extremly small amount of data, and the scarcity of high resolution DEM of these regions.

Despite advances in terrain generation, existing methods struggle with user-controlled design of specific island shapes and achieving realism without real data. Coral reef islands exemplify this gap: we lack datasets to directly train deep models, and purely procedural methods require expert tuning to mimic their features.

To address these issues, we use procedural generation as an initial step in our approach as a means to efficiently create a large and diverse set of training examples for a learning-based model. Each synthetic example is represented by a terrain height field and a corresponding semantic label map that marks different regions, providing structured input-output pairs for the learning stage as presented in \cref{fig:coral-island_cGAN-examples}.

We trained a conditional Generative Adversarial Network (cGAN) as the core of our learning-based approach (\cite{Mirza2014,Isola2017}). A cGAN is a type of deep learning model that learns to generate realistic data based on an input condition or context. In our case, the cGAN takes as input the semantic label map of an island generated by the procedural step and learns to produce a realistic island height field that matches this layout. By training on the many examples from the procedural generator, the cGAN captures the subtle terrain features and variations characteristic of coral reef islands, going beyond what hard-coded procedural rules can achieve thanks to the application of data augmentation. The cGAN model can be used on its own to generate new island terrains with simplified and more intuitive user inputs through digital drawing, and the model will generate a realistic island terrain accordingly.

The key contributions are as follows: 1) a novel sketch-based procedural algorithm for shaping island terrains from top and profile views, 2) The training of a deep learning model on synthetic data derived from procedural rules, serving as an abstraction layer that hides underlying complexity, 3) A demonstration that the cGAN approach tolerates imprecise, low-detail user input sketches, broadening usability, without the need for cutting-edge network architectures, and 4) An insight that procedural generation remains essential to produce training data in data-sparse domains such as coral reef islands. 
These contributions collectively show a pathway to blend user-driven design with learning-based generation in terrain modeling.