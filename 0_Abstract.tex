
\begin{abstract}
We propose a procedural method for generating single volcanic islands with coral reefs using user sketching from two projections: a top view, which defines the island's shape, and a profile view, which outlines its elevation. These projections, commonly used in geological and remote sensing domains, are complemented by a user-defined wind field, applied as a distortion field to deform the island's shape, mimicking the effects of wind and waves on the long term and enabling finer user control. We then model the growth of coral on the island and its surrondings to construct the reef following biological observations. Based on these inputs, our method generates a height field of the island. Our method creates a large variety of island models to compose a dataset used for training a conditional Generative Adversarial Network (cGAN). By applying data augmentation, the cGAN allows for even greater variety in the generated islands, providing users with higher freedom and intuitive controls over the shape and structure of the final output.
\end{abstract}

\begin{keywords}
Procedural modeling, Terrain generation, cGAN, Coral reef, Sketch-based interface
\end{keywords}